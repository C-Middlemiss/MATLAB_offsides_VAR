%%%%%%%%%%%%%%%%%%%%%%%%%%%%%%%%%%%%%%%%%%%%%%%%%%%%%%%%%%%%%%%%%%%%%%%%%%%%%%%%
%2345678901234567890123456789012345678901234567890123456789012345678901234567890
%        1         2         3         4         5         6         7         8

\documentclass[letterpaper, 10 pt, conference]{ieeeconf}  % Comment this line out
																													% if you need a4paper
%\documentclass[a4paper, 10pt, conference]{ieeeconf}      % Use this line for a4
																													% paper

\IEEEoverridecommandlockouts                              % This command is only
																													% needed if you want to
																													% use the \thanks command
\overrideIEEEmargins
% See the \addtolength command later in the file to balance the column lengths
% on the last page of the document



% The following packages can be found on http:\\www.ctan.org
%\usepackage{graphics} % for pdf, bitmapped graphics files
%\usepackage{epsfig} % for postscript graphics files
%\usepackage{mathptmx} % assumes new font selection scheme installed
%\usepackage{times} % assumes new font selection scheme installed
%\usepackage{amsmath} % assumes amsmath package installed
%\usepackage{amssymb}  % assumes amsmath package installed

\title{\LARGE \bf
VAR System for Recognizing Offsides in Soccer
}


\author{Calum Middlemiss, Andrew Allen, and Colin Panarra% <-this % stops a space
}


\begin{document}



\maketitle
\thispagestyle{empty}
\pagestyle{empty}


%%%%%%%%%%%%%%%%%%%%%%%%%%%%%%%%%%%%%%%%%%%%%%%%%%%%%%%%%%%%%%%%%%%%%%%%%%%%%%%%
\begin{abstract}

This semester, working on the identification of offsides in soccer through an automated Video Assistant referee system has culminated in an imperfect network of methods that if properly linked could allow for a functional VAR system. In order to identify instances of offsides through video we broke the problem down into key segments that, if executed sequentially, would allow us to determine when a player was in an offsides position at the time of a forward pass. These segments are object tracking of the soccer ball, measuring the balls velocity to determine a "forward" pass, isolating the still frame from the video at the time of the start of the forward pass, segmenting this frame to identify key features, classifying these features, and determining whether any player was in an offside position. 

\end{abstract}


%%%%%%%%%%%%%%%%%%%%%%%%%%%%%%%%%%%%%%%%%%%%%%%%%%%%%%%%%%%%%%%%%%%%%%%%%%%%%%%%
\section{INTRODUCTION}



\section{Procedure}

\subsection{object tracking of the soccer ball}

\subsection{measuring the balls velocity to determine a "forward" pass}

\subsection{isolating the still frame from the video at the time of the start of the forward pass}

\subsection{segmenting this frame to identify key features}

\subsection{classifying features}

\subsection{determining whether any player was in an offside position}

\section{Results and Discussion}

\section{Present Issues}

\section{Future Work}

\addtolength{\textheight}{-12cm}   % This command serves to balance the column lengths
																	% on the last page of the document manually. It shortens
																	% the textheight of the last page by a suitable amount.
																	% This command does not take effect until the next page
																	% so it should come on the page before the last. Make
																	% sure that you do not shorten the textheight too much.

%%%%%%%%%%%%%%%%%%%%%%%%%%%%%%%%%%%%%%%%%%%%%%%%%%%%%%%%%%%%%%%%%%%%%%%%%%%%%%%%

\begin{thebibliography}{99}

\bibitem{c1} Direkoglu, Cem, et al. “Player detection in field sports.” Machine Vision and 
Applications, vol. 29, no. 2, May 2017, pp. 187–206., doi:10.1007/s00138-017-0893-8.

\bibitem{c2}Hashimoto, Sadatsugu, and Shinji Ozawa. “A System for Automatic Judgment of 
Offsides in Soccer Games.” 2006 IEEE International Conference on Multimedia and Expo, 2006, doi:10.1109/icme.2006.262924.

\bibitem{c3}Yang, Ying, and Danyang Li. “Robust player detection and tracking in broadcast 
soccer video based on enhanced particle filter.” Journal of Visual 
Communication and Image Representation, vol. 46, 2017, pp. 81–94., doi:10.1016/j.jvcir.2017.03.008.

\end{thebibliography}
\end{document}
